\documentclass[aps,rmp, onecolumn]{revtex4}
%\documentclass[a4paper,10pt]{scrartcl}
%\documentclass[aps,rmp,twocolumn]{revtex4}

\usepackage[utf8]{inputenc}
\usepackage{amsmath,graphicx}
\usepackage{color}
%\usepackage{cite}

\newcommand{\bq}{\begin{equation}}
\newcommand{\eq}{\end{equation}}
\newcommand{\bn}{\begin{eqnarray}}
\newcommand{\en}{\end{eqnarray}}
\newcommand{\Vadim}[1]{{\color{blue}Vadim: #1}}
\newcommand{\Richard}[1]{{\color{red}Richard: #1}}
\newcommand{\gene}[1]{{\it #1}}
\newcommand{\mat}[1]{{\bf #1}}
\newcommand{\vecb}[1]{{\bf #1}}
\newcommand{\abet}{\mathcal{A}}
\newcommand{\eqp}{p}
\newcommand{\LH}{\mathcal{L}}
\pdfinfo{%
  /Title    ()
  /Author   ()
  /Creator  ()
  /Producer ()
  /Subject  ()
  /Keywords ()
}


\begin{document}

\title{Site-specific substitution model inference with iterative tree likelihood maximization.}
\author{Vadim Puller$^{1,2,3}$ Pavel Sagulenko$^{1}$, Richard Neher$^{1,2,3}$}
\affiliation{$^{1}$Max Planck Institute for Developmental Biology, 72076 T\"ubingen, Germany\\
$^{2}$Biozentrum, University of Basel, Klingelbergstrasse 50/70, 4056 Basel, Switzerland\\
$^{3}$SIB Swiss Institute of Bioinformatics, Basel, Switzerland}

\date{\today}

\maketitle

Most models of sequence evolution express the probability that sequence $\vec{s}$ evolved from sequence $\vec{r}$ in time $t$ as
\bq
P(\vec{s}\leftarrow \vec{r},t,\mat{Q}) = \prod_{a=1}^L \left(e^{\mat{Q}^{a} t}\right)_{s_{a},r_{a}}
\label{eq:P_prob}
\eq
where $\mat{Q}^{a}$ is the substitution matrix governing evolution at site $i$, and $s_{a}$ and $r_{a}$ are the sequence states at position $i$.
The product runs over all $L$ sites $a$ and amounts to assuming that different sites of the sequence evolve independently.

In absence of recombination, homologous sequences are related by a tree and the likelihood of observing an alignment $\mat{A}=\{\vec{s^k}, k=1..n\}$ conditional on the tree $T$ and the substitution model $\mat{Q}^{a}$ can be written in terms of propagators defined in Eq.~\ref{eq:P_prob}.
It is helpful to express this likelihood as product of sequence propagators defined in Eq.~\ref{eq:P_prob} between sequences at the ends of each branch in the tree (implicitly assuming that on different branches is independent and follows the same time reversible model).
Unknown sequences of internal nodes $\{\vec{x}\}$ need to be summed over and the likelihood can be expressed as
\begin{equation}
	\LH(\mat{A}|T,\mat{Q}) = \sum_{\{\vec{x}\}}\prod_{a=1}^L \eqp^{a}_{x^{0}_a} \prod_{k\in T}P(\vec{s}_{k_c}\leftarrow \vec{s}_{k_p},t,\mat{Q}) \ ,
	\label{eq:LH}
\end{equation}
where $\vec{s}_{k_c}$ and $\vec{s}_{k_p}$ are the child and parent sequences of branch $k$, respectively, and the factor $\prod \eqp^{a}_{x^{0}_a}$ is the product of the probabilities of the root sequence $x^{0}_a$ over all positions $i$.
The probabilities $\vec{\eqp}^{a}$ are the equiblirium probabilities of the substitution model at position $a$.
The latter ensures that the likelihood is insensitive to a particular choice of the tree root.

The likelihood can be computed efficiently using standard dynamic programming techniques.
Nevertheless, it requires $\mathcal{O}(n\times L \times q^2)$ operations (where $q$ is the size of the alphabet $\abet$) and optimizing it with respect to a large number of parameters is costly.
Our goal here is to infer site-specific substitution models using a computationally efficient iterative procedure.

We will parameterize the substitution model as follows:
We describe the substitution process using site-specific GTR
(general time-reversible) model, parametrized as:
\begin{eqnarray}
Q^{a}_{ij} &=& \mu^{a}\eqp^{a}_{i} W_{ij} \textrm{ for } i\neq j,\nonumber \\
Q^{a}_{ii} &=& -\sum_k Q^{a}_{ki}
\label{eq:Qij}
\end{eqnarray}
where $W_{ij}$ is a symmetric matrix and the second equation ensures conservation of probability.
In addition, we require $\sum_i \eqp^{a}_i = 1$ and $\sum_{a=1}^L\sum_{i\neq j}W_{ij}p^{a}_ip^{a}_j=L$ to ensure that the average rate per site is $\mu^{a}$.

For each branch $k$ in the tree we can compute the likelihoods of the parent and child subtrees obtained by this branch conditional on the sequence at either end of the branch.
This, in turn, allows us to compute the marginal distribution $\rho^{a}_i$ and $\sigma^{a}_i$ of character $i$ at position $a$ of root sequence of the parent and child subtrees, respectively.

Using the product from of the likelihood, we can now write down compact expressions for the derivative of the likelihood and the with respect to the substitution model.
\begin{equation}
	\frac{\partial \log \LH}{\partial \mu^{b}} = \sum_{k\in T} \frac{t_k}{\mu^{b}}\frac{\vec{\sigma}^{k,b} \mat{Q}^{b} e^{\mat{Q}^{b}t_k} \vec{\rho}^{k,b}}{\vec{\sigma}^{k,b} e^{\mat{Q}^{b}t_k} \vec{\rho}^{k,b}}
\end{equation}
Analogously, we find for the derivaties wrt to $\eqp^{k,b}_{i}$
\begin{equation}
\frac{\partial \log \LH}{\partial \eqp^{b}_i}= \sum_{k\in T} \frac{t_k}{\eqp^{b}_i}\frac{\sum_{h,j}\sigma^{k,b}_i Q^{k,b}_{ih} e^{\mat{Q}^{b}t_k}_{hj} \rho^{k,b}_j}{\vec{\sigma}^{k,b} e^{\mat{Q}^{b}t_k} \vec{\rho}^{k,b}} + \frac{\sigma^{0,b}_i}{\eqp^b_i}
\end{equation}
where $\sigma^{0,b}_i$ corresponds to state of the root sequence at position $b$.
The corresponding derivative for the transition matrix $W_{rs}$ is
\begin{equation}
	\frac{\partial \log \LH}{\partial W_{ij}}= \sum_{k\in T} \frac{t_k}{W_{ij}} \sum_{a=1}^L\frac{\sum_{h}\sigma^{k,b}_i Q^{k,b}_{ih} e^{\mat{Q}^{b}t_k}_{hj} \rho^{k,b}_j}{\vec{\sigma}^{k,b} e^{\mat{Q}^{b}t_k} \vec{\rho}^{k,b}}
\end{equation}
The at an extremum, all these derivatives have to vanish.
In addition we need to impose the normalization constraints which is done in the form of Lagrange multipliers such that the final conditions are
\begin{eqnarray}
	0 &=& \sum_{k\in T} t_k\frac{\vec{\sigma}^{k,b} \mat{Q}^{b} e^{\mat{Q}^{b}t} \vec{\rho}^{k,b}}{\vec{\sigma}^{k,b} e^{\mat{Q}^{b}t_k} \vec{\rho}^{k,b}} \nonumber \\
    0 &=& 	\sum_{k\in T} t_k\frac{\sum_{h,j}\sigma^{k,b}_i Q^{k,b}_{ih} e^{\mat{Q}^{b}t_k}_{hj} \rho^{k,b}_j}{\vec{\sigma}^{k,b} e^{\mat{Q}^{b}t} \vec{\rho}^{k,b}} + \sigma^{0,b}_i + \lambda^{b}\eqp^b_i + \kappa \sum_j W_{ij}\eqp_i^{b}\eqp_j^{b} \nonumber\\
	0 &=& \sum_{k\in T} t_k \sum_{a=1}^L\frac{\sum_{h}\sigma^{k,b}_i Q^{k,b}_{ih} e^{\mat{Q}^{b}t_k}_{hj} \rho^{k,b}_j}{\vec{\sigma}^{k,b} e^{\mat{Q}^{b}t_k} \vec{\rho}^{k,b}} + \kappa W_{ij}\sum_{b=1}^L \eqp^b_i\eqp^b_j
\end{eqnarray}
Summing the second and third equation over $i$ and $i,j$, respectively, and substituting the first equation, we find $\kappa=0$ and $\lambda^a=1$.

To arrive at an iterative scheme to solve these equations, it is helpful to split the individuals terms into \emph{diagonal} contributions (the case in which the sequence does not change along a branch) and \emph{off-diagonal} contributions (substitutions happened on that branch).
We will first consider the limit of short branches were we can approximate $e^{\mat{Q}^at}=\mat{I} + \mat{Q}^at$ and evaluate to leading order
\begin{equation}
\begin{split}
t_k\frac{\vec{\sigma}^{k,b} \mat{Q}^{b} e^{\mat{Q}^{b}t} \vec{\rho}^{k,b}}{\vec{\sigma}^{k,b} e^{\mat{Q}^{b}t_k} \vec{\rho}^{k,b}} \approx &	t_k \frac{-\sum_{i} \mu^b\sum_h p_h^b W_{hi} \sigma^{k,b}_i \rho^{k,b}_i + \sum_{i, j\neq i} \mu^b\sigma^{k,b}_i p_i^bW_{ij} \rho^{k,b}_j}{\sum_{i} \sigma^{k,b}_i \rho^{k,b}_i + \sum_{i,j\neq i} \mu^b \sigma^{k,b}_i \eqp_i^b W_{ij}t_k \rho^{k,b}_j}\\
\end{split}
\end{equation}
In the limit of short branches (or if ancestral states are known), $\sigma^{k,b}_i$ and $\rho^{k,b}_i$ are either $0$ or $1$ such that only one of the two terms in the numerator and denominator contribute.
In case $\sigma^{k,b}_i=\rho^{k,b}_i$ (no substitution on that branch), the denominator is 1 and the numerator $- \mu^b \sum_{i,h}p_h^b W_{hi}t_k$.
If a substitution happened, numerator and denominator cancel with result 1.
After summing over all branches, the condition for the mutation rate simplifies to
\begin{equation}
	\mu^b \sum_i \eqp_i^b W_{ij}T_j^b = \sum_{i\neq j} n_{ij}
\end{equation}
where $T_j^b$ is the sum of all $t_k$ were the sequence is in state $j$.



\bibliography{GTRbib}

\end{document}
